\documentclass{report}

\usepackage[utf8]{inputenc}
\usepackage{amsmath}
\usepackage{graphicx}
\graphicspath{{"./images/"}}

\textwidth 14cm
\hoffset -1cm


\title{\Huge Simulador de Planificación de Procesos}
\author{David Castaño de la Mota \\ DNI: 07246190V \\ Email: davidcastanomota@gmail.com\\Madrid}
\date{21 de Diciembre de 2020}


\begin{document}
\maketitle

\tableofcontents{}


\chapter{Introduction}
    \section{Overview}
        This document describes the OS Planning Simulator tool developed under the scope of the "" PFG, and the details about the faced problems and the solutions applied to solve them.
    \section{Objectives}
        The aim of this project is to develop a graphical simulator of OS Process Scheduler. The tool includes the most usual planning algorithms that are implemented in known Operative Systems.\\
        The tool allows the user to analyze, in a step-by-step mode, the logic of the processes execution. The visualization of the concurrency, problems such as deadlocks, and CPU use statistics can provide the user with an intuitive representation of the different situations.
    \section{Algorithms}
        This chapter describes the different algorithms used through the history of the computing science. These methods are available in the developed simulation tool, makin possible the comparison of their efficiency.
        \subsection{FCFS (Fisrt come First server}
        \subsection{SJF (Short Job First)}
        \subsection{Round Robin}
        \subsection{Priorities Algorithms}

    \section{Other Simulators}
        \subsection{A Neural Network Playground}
            This simulator is availaible in the web (https://playground.tensorflow.org/)
            \begin{center}
                \includegraphics[width=8cm]{Playground.png}
            \end{center}  
\chapter{Application Development}

\chapter{References}
\end{document}
